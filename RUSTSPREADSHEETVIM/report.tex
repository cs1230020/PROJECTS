\documentclass{article}
\usepackage{graphicx} % Required for inserting images
\usepackage{hyperref}

\title{Explains design and software architecture}
\author{
  Subhash Malaviya (2023CS10749) \\
  Soumodeep Chatterjee (2023CS50299) \\ 
  Shashank Kumar (2023CS10020)
}
\date{April 2025}

\begin{document}

\maketitle

\section*{Team Members}

\noindent
\parbox{\textwidth}{
  SUBHASH MALAVIYA (2023CS10749)\\
  SOUMODEEP CHATTERJEE (2023CS50299)\\
  SHASHANK KUMAR (2023CS10020)
}

\section*{GitHub Repository}

The complete source code and documentation are available at: 

\noindent
\url{https://github.com/Pacify-14/Rust-Lab-COP-290}

\section{Why Proposed Extension Could Not Be Fully Implemented}

While the core Vim-like terminal and GUI-based spreadsheet functionality was successfully implemented, several proposed extensions could not be realized due to time and complexity constraints. In particular:
\begin{itemize}
  \item \textbf{Chart and Data Visualization} features like statistical plots, time series, and correlation graphs were not implemented.
  \item \textbf{Custom mathematical functions}, hierarchical expression parsing, and dynamic formula libraries were left out.
  \item \textbf{Smart data management features} such as filtering by Boolean conditions or dynamic query execution were not developed.
  \item \textbf{Performance optimizations} like parallel formula evaluation and lazy evaluation were planned but not implemented.
  \item \textbf{Automated testing frameworks} including property-based testing were proposed but not incorporated.
\end{itemize}

\section{Could We Implement Extra Extensions Over and Above the Proposal?}

Yes, some enhancements were implemented that went beyond the originally proposed scope:
\begin{itemize}
  \item A fully-functional \textbf{GUI-based interface} using the \texttt{egui} framework in addition to the terminal interface.
  \item \textbf{Visual selection support} in the GUI and TUI for range operations like copying and pasting ranges.
  \item \textbf{File format flexibility}, including support for CSV, TSV, and a custom SS format with formula preservation.
\end{itemize}

\section{Primary Data Structures Used}

\begin{itemize}
  \item \texttt{Vec<Vec<cell>>} – A 2D vector used as the spreadsheet grid.
  \item \texttt{cell} struct – Stores individual cell state including value, formula, and error flag.
  \item \texttt{EditorState} – Maintains UI mode, cursor position, selection range, and buffer states.
  \item \texttt{ClipboardContent} enum – Encapsulates cell, row, column, and range data for copy-paste operations.
  \item \texttt{DAGNode} and \texttt{Node} – Represent the dependency graph for formula evaluation.
\end{itemize}

\section{Interfaces Between Software Modules}

The architecture is modularized as follows:
\begin{itemize}
  \item \textbf{core logic} (formula parsing, evaluation) in \texttt{main.rs}
  \item \textbf{editor state and modes} in \texttt{editor.rs}
  \item \textbf{terminal interface} in \texttt{ui.rs}
  \item \textbf{GUI interface} in \texttt{egui\_ui.rs}
  \item \textbf{command execution} in \texttt{commands.rs}
\end{itemize}

\section{Approaches for Encapsulation}

Encapsulation was maintained by:
\begin{itemize}
  \item Defining stateful structs like \texttt{EditorState} with controlled access to internal fields.
  \item Using enums like \texttt{Mode} and \texttt{ClipboardContent} for explicit mode/state management.
  \item Separating interface logic from core logic and data manipulation routines.
\end{itemize}

\section{Justification of Design}

This design enables:
\begin{itemize}
  \item \textbf{Robust terminal and GUI interaction} via mode-specific input handling.
  \item \textbf{High extensibility}, allowing new commands and UI features to be integrated with minimal change.
  \item \textbf{User efficiency}, leveraging familiar Vim-like controls.
  \item \textbf{Clean separation of concerns}, which improves code maintainability and testability.
\end{itemize}

\section{Modifications to the Initial Design}

The following deviations were made from the original proposal:
\begin{itemize}
  \item Inclusion of the \texttt{egui}-based graphical interface, which was not originally planned.
  \item Reduction of scope in terms of visualization and advanced querying features to prioritize core spreadsheet stability.
  \item Extended copy-paste functionality to support entire rows and columns, beyond individual cells.
\end{itemize}

\section{Code Architecture and Workflow}

\subsection{Overview of Functionality}

The Vim-like spreadsheet is a terminal and GUI-based editor developed in Rust. It supports multiple editing modes, efficient keyboard navigation, expression evaluation, and formula-based cell dependencies. The interface mimics \texttt{vim} behavior, offering modes like \texttt{Normal}, \texttt{Insert}, \texttt{Visual}, and \texttt{Command} for power users.

\subsection{Core Module Responsibilities}

\begin{itemize}
  \item \textbf{\texttt{main.rs}}: Entry point and core logic. It handles formula parsing, topological evaluation using a dependency graph (DAG), scroll commands, and command-line input.
  \item \textbf{\texttt{editor.rs}}: Manages editor state and mode transitions. It defines the \texttt{EditorState} struct and modes (Insert, Normal, Visual, Command). Also handles edit buffer and clipboard.
  \item \textbf{\texttt{ui.rs}}: Implements the terminal-based UI using the \texttt{crossterm} crate. It renders the spreadsheet grid, cursor, and status bar, and processes keyboard input based on the current mode.
  \item \textbf{\texttt{egui\_ui.rs}}: Provides an optional GUI interface using \texttt{eframe} and \texttt{egui}. It supports mouse and keyboard events, colored mode banners, and cell-based interactions.
  \item \textbf{\texttt{commands.rs}}: Parses and executes \texttt{:commands} in Command mode, including \texttt{:w}, \texttt{:q}, \texttt{:e}, search/replace, and batch operations.
  \item \textbf{\texttt{mod.rs}}: Central re\-export module that links together the \texttt{commands}, \texttt{editor}, \texttt{ui}, and \texttt{egui\_ui} modules under the \texttt{vim\_mode} namespace.
\end{itemize}

\subsection{Data Flow and Evaluation}

The spreadsheet grid is represented as a \texttt{Vec<Vec<cell>>}, where each \texttt{cell} stores:
\begin{itemize}
  \item An optional formula string
  \item An evaluated integer value
  \item An error flag (1 if invalid, 0 otherwise)
\end{itemize}

When a formula is entered, it is parsed and stored in the cell. During evaluation, a dependency graph (\texttt{Vec<DAGNode>}) is built. A topological sort ensures cells are evaluated in correct order, respecting dependencies and propagating errors.

Supported formulas include:
\begin{itemize}
  \item Arithmetic: \texttt{A1 + B2}, \texttt{3 * C3}
  \item Aggregates: \texttt{SUM(A1:A5)}, \texttt{AVG(B1:B3)}
  \item Functions: \texttt{SLEEP(3)}, \texttt{SLEEP(B2)}
\end{itemize}

\subsection{Interaction Flow}

\begin{enumerate}
  \item \textbf{Startup:} The program initializes the grid and starts in Normal mode.
  \item \textbf{Navigation:} Users move the cursor using \texttt{h, j, k, l}.
  \item \textbf{Editing:} Pressing \texttt{i} enters Insert mode, where formulas or values can be typed.
  \item \textbf{Command execution:} Pressing \texttt{:} enters Command mode. Commands like \texttt{:wq}, \texttt{:e filename}, or \texttt{:A1} are parsed and executed.
  \item \textbf{Visual selection:} Pressing \texttt{v} activates Visual mode to select ranges for copy/paste.
  \item \textbf{Evaluation:} After edit or file load, the sheet is re-evaluated using a DAG-based topological traversal.
\end{enumerate}

\subsection{Formula Evaluation Example}

Given:
\begin{verbatim}
A1 = 5
A2 = A1 + 3
A3 = SUM(A1:A2)
\end{verbatim}

The evaluation order will be:
\begin{itemize}
  \item \texttt{A1}: directly assigned value 5
  \item \texttt{A2}: depends on \texttt{A1}, evaluates to 8
  \item \texttt{A3}: range sum of A1 and A2 = 13
\end{itemize}

\subsection{Clipboard Functionality}

Copying is done using \texttt{y}, pasting with \texttt{p}. Clipboard supports:
\begin{itemize}
  \item Single cells
  \item Rows and columns
  \item Arbitrary rectangular ranges
\end{itemize}

These are stored using the \texttt{ClipboardContent} enum.

\subsection{Error Handling}

\texttt{ERR} is displayed in cells with:
\begin{itemize}
  \item Invalid formulas
  \item Circular dependencies
  \item References to error cells
\end{itemize}

\subsection{Search and Replace}

Searches can be performed using \texttt{/pattern} and \texttt{?pattern}. Matches are navigated using \texttt{n/N}. Replace is done using \texttt{:s/old/new/g}.

\subsection{Extensibility}

The modular design makes it easy to:
\begin{itemize}
  \item Add new commands in \texttt{commands.rs}
  \item Extend formula parsing in \texttt{main.rs}
  \item Introduce new modes via \texttt{editor.rs}
  \item Improve visualization using \texttt{egui\_ui.rs}
\end{itemize}


\end{document}

